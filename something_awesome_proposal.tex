\documentclass{article}
\usepackage[utf8]{inputenc}

\title{Something Awesome Proposal:\\(Mostly) Weekly Crypto Challenge}
\author{Youhan Cheery (z3417483)}
\date{16 March 2017}

\begin{document}

\maketitle
\section{Introduction}
\begin{itemize}
    \item Completing a weekly cryptography challenge found on \textit{http://cryptopals.com/}
    \item I have some theoretical understanding of cryptography (through MATH3411 - Information, Codes and Ciphers). However I have no experience actually applying the theory that I learned.
    \item I would like to undertake the weekly challenges, of which there are about 6-7 sub-challenges in each of these.
    \item The weekly topics include:
    \begin{enumerate}
        \item Basics
        \item Block cryptography
        \item Block and stream cryptography
        \item Stream crypto and randomness
        \item Diffie-Hellman
        \item RSA and DSA
        \item Hashes
        \item Abstract algebra
    \end{enumerate}
    \item I would then like to compile the work that it takes to solve these problems in a separate report for future students' use.
    \item As I am currently working on a thesis, I imagine I will go through periods of inactivity depending on the state of my thesis. To prevent going through long periods of inactivity on the cryptography challenges, I will outline the theory of the modules involved in the report prior to undertaking the challenge. I don't think this in itself is trivial, but I'm fairly certain it will take a lot less time than the challenges. 
\end{itemize}
\section{Plan}
Assuming the project will begin from week 4, that leaves me approximately 10 weeks (including the break week) for the due date. Cryptopals claims that the difficult increases further down the line, which on the surface seems true given the topic titles. As such, having a 2 week buffer to handle more complex challenges seems reasonable.\\

I am not under the impression I will be able to complete everything. However I will definitely attempt to, and will blog as I go along. 
\end{document}
